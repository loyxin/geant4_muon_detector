%!TeX encoding = UTF-8
%!TeX program = xelatex
\documentclass[notheorems, aspectratio=54]{beamer}
% aspectratio: 1610, 149, 54, 43(default), 32

\usepackage{latexsym}
\usepackage{amsmath,amssymb}
\usepackage{mathtools}
\usepackage{color,xcolor}
\usepackage{graphicx}
\usepackage{algorithm}
\usepackage{amsthm}
\usepackage{lmodern} % 解决 font warning
\usepackage[UTF8]{ctex}


\usepackage{lipsum} % To generate test text 
\usepackage{ulem} % 下划线,波浪线

\usepackage{listings} % display code on slides; don't forget [fragile] option after \begin{frame}


\usepackage{enumerate}
\usepackage{caption}
\usepackage{subfigure}

% ----------------------------------------------
% tikx
\usepackage{framed}
\usepackage{tikz}
\usepackage{pgf}
\usetikzlibrary{calc,trees,positioning,arrows,chains,shapes.geometric,%
    decorations.pathreplacing,decorations.pathmorphing,shapes,%
    matrix,shapes.symbols}
\pgfmathsetseed{1} % To have predictable results
% Define a background layer, in which the parchment shape is drawn
\pgfdeclarelayer{background}
\pgfsetlayers{background,main}

% define styles for the normal border and the torn border
\tikzset{
  normal border/.style={orange!30!black!10, decorate, 
     decoration={random steps, segment length=2.5cm, amplitude=.7mm}},
  torn border/.style={orange!30!black!5, decorate, 
     decoration={random steps, segment length=.5cm, amplitude=1.7mm}}}

% Macro to draw the shape behind the text, when it fits completly in the
% page
\def\parchmentframe#1{
\tikz{
  \node[inner sep=2em] (A) {#1};  % Draw the text of the node
  \begin{pgfonlayer}{background}  % Draw the shape behind
  \fill[normal border] 
        (A.south east) -- (A.south west) -- 
        (A.north west) -- (A.north east) -- cycle;
  \end{pgfonlayer}}}

% Macro to draw the shape, when the text will continue in next page
\def\parchmentframetop#1{
\tikz{
  \node[inner sep=2em] (A) {#1};    % Draw the text of the node
  \begin{pgfonlayer}{background}    
  \fill[normal border]              % Draw the ``complete shape'' behind
        (A.south east) -- (A.south west) -- 
        (A.north west) -- (A.north east) -- cycle;
  \fill[torn border]                % Add the torn lower border
        ($(A.south east)-(0,.2)$) -- ($(A.south west)-(0,.2)$) -- 
        ($(A.south west)+(0,.2)$) -- ($(A.south east)+(0,.2)$) -- cycle;
  \end{pgfonlayer}}}

% Macro to draw the shape, when the text continues from previous page
\def\parchmentframebottom#1{
\tikz{
  \node[inner sep=2em] (A) {#1};   % Draw the text of the node
  \begin{pgfonlayer}{background}   
  \fill[normal border]             % Draw the ``complete shape'' behind
        (A.south east) -- (A.south west) -- 
        (A.north west) -- (A.north east) -- cycle;
  \fill[torn border]               % Add the torn upper border
        ($(A.north east)-(0,.2)$) -- ($(A.north west)-(0,.2)$) -- 
        ($(A.north west)+(0,.2)$) -- ($(A.north east)+(0,.2)$) -- cycle;
  \end{pgfonlayer}}}

% Macro to draw the shape, when both the text continues from previous page
% and it will continue in next page
\def\parchmentframemiddle#1{
\tikz{
  \node[inner sep=2em] (A) {#1};   % Draw the text of the node
  \begin{pgfonlayer}{background}   
  \fill[normal border]             % Draw the ``complete shape'' behind
        (A.south east) -- (A.south west) -- 
        (A.north west) -- (A.north east) -- cycle;
  \fill[torn border]               % Add the torn lower border
        ($(A.south east)-(0,.2)$) -- ($(A.south west)-(0,.2)$) -- 
        ($(A.south west)+(0,.2)$) -- ($(A.south east)+(0,.2)$) -- cycle;
  \fill[torn border]               % Add the torn upper border
        ($(A.north east)-(0,.2)$) -- ($(A.north west)-(0,.2)$) -- 
        ($(A.north west)+(0,.2)$) -- ($(A.north east)+(0,.2)$) -- cycle;
  \end{pgfonlayer}}}

% Define the environment which puts the frame
% In this case, the environment also accepts an argument with an optional
% title (which defaults to ``Example'', which is typeset in a box overlaid
% on the top border
\newenvironment{parchment}[1][Example]{%
  \def\FrameCommand{\parchmentframe}%
  \def\FirstFrameCommand{\parchmentframetop}%
  \def\LastFrameCommand{\parchmentframebottom}%
  \def\MidFrameCommand{\parchmentframemiddle}%
  \vskip\baselineskip
  \MakeFramed {\FrameRestore}
  \noindent\tikz\node[inner sep=1ex, draw=black!20,fill=white, 
          anchor=west, overlay] at (0em, 2em) {\sffamily#1};\par}%
{\endMakeFramed}

% ----------------------------------------------

\mode<presentation>{
    \usetheme{CambridgeUS}
    % Boadilla CambridgeUS
    % default Antibes Berlin Copenhagen
    % Madrid Montpelier Ilmenau Malmoe
    % Berkeley Singapore Warsaw
    \usecolortheme{beaver}
    % beetle, beaver, orchid, whale, dolphin
    \useoutertheme{infolines}
    % infolines miniframes shadow sidebar smoothbars smoothtree split tree
    \useinnertheme{circles}
    % circles, rectanges, rounded, inmargin
}
% 设置 block 颜色
\setbeamercolor{block title}{bg=red!30,fg=white}

\newcommand{\reditem}[1]{\setbeamercolor{item}{fg=red}\item #1}

% 缩放公式大小
\newcommand*{\Scale}[2][4]{\scalebox{#1}{\ensuremath{#2}}}

% 解决 font warning
\renewcommand\textbullet{\ensuremath{\bullet}}

% ---------------------------------------------------------------------
% flow chart
\tikzset{
    >=stealth',
    punktchain/.style={
        rectangle, 
        rounded corners, 
        % fill=black!10,
        draw=white, very thick,
        text width=6em,
        minimum height=2em, 
        text centered, 
        on chain
    },
    largepunktchain/.style={
        rectangle,
        rounded corners,
        draw=white, very thick,
        text width=10em,
        minimum height=2em,
        on chain
    },
    line/.style={draw, thick, <-},
    element/.style={
        tape,
        top color=white,
        bottom color=blue!50!black!60!,
        minimum width=6em,
        draw=blue!40!black!90, very thick,
        text width=6em, 
        minimum height=2em, 
        text centered, 
        on chain
    },
    every join/.style={->, thick,shorten >=1pt},
    decoration={brace},
    tuborg/.style={decorate},
    tubnode/.style={midway, right=2pt},
    font={\fontsize{10pt}{12}\selectfont},
}
% ---------------------------------------------------------------------

% code setting
\lstset{
    language=C++,
    basicstyle=\ttfamily\footnotesize,
    keywordstyle=\color{red},
    breaklines=true,
    xleftmargin=2em,
    numbers=left,
    numberstyle=\color[RGB]{222,155,81},
    frame=leftline,
    tabsize=4,
    breakatwhitespace=false,
    showspaces=false,               
    showstringspaces=false,
    showtabs=false,
    morekeywords={Str, Num, List},
}

% ---------------------------------------------------------------------

%% preamble
\title[muon detector simulation by using GEANT4]{muon detector simulation by using GEANT4}
% \subtitle{The subtitle}
\author{罗鑫}
\institute[SYSU]{luox46@mail2.sysu.edu.cn}

% -------------------------------------------------------------

\begin{document}

%% title frame
\begin{frame}
    \titlepage
\end{frame}

%% normal frame
\section{Introduction to geant4}
\subsection{}
\begin{frame}
  \frametitle{Introduction to geant4}
  Geant4是由欧洲核子中心(CERN)和日本高能物理中心(KEK)主导开发的蒙特卡罗辐射输运计算通用程序包,主要应用在高能物理领域,可方便模拟强相互作用、弱相互作用等高能、超高能物理过程。
\begin{block}{什么是蒙特卡罗}
  比如计算圆周率:
  \begin{itemize}
    \item 以直画圆
    \item $\pi$的理论展开公式
    \item 。。。。
  \end{itemize}
  简单的一个算法:
  \begin{itemize}
    \item 撒点
  \end{itemize}
\end{block}

\begin{block}{}
优点
\end{block}
\end{frame}
\section{Introduction to $\mu$}
\subsection{}

\begin{frame}
缪子是一种与电子相似的基本粒子,符号 $\mu^-$, 它带有 1 单位负电荷,自旋为 1/2,质量为 105 $MeV/c^2$. 缪子的反粒子是 $\mu^+$,拥有 1 单位正电荷.

地球上绝大部分自然生成的缪子都由宇宙
线中的 $\pi$ 介子产生($\pi^- \rightarrow \mu^- + \bar{\nu_u},\ \pi^+\rightarrow \mu^+ + \nu_\mu$ ).大多数缪子在海平面以上 15km 处产生.

缪子的平均寿命约为$2.2\mu s$, 准确值为 $2.1969811(22) 10^{−6} s$,缪子的衰变方式:
\begin{align}
\mu^- \rightarrow e^- + \nu_\mu + \bar\nu_\mu \\
\mu^+ \rightarrow e^+ + \bar\nu_\mu + \nu_\mu
\end{align}

\end{frame}
\section{Simulation}
\subsection{}

\begin{frame}
\center{
\includegraphics[height=150pt]{../pic/muondetect.png}
}
\begin{block}{Video}
\end{block}
\end{frame}

\section{result}
\subsection{detector}
\begin{frame}
\begin{figure}[htbp]
\centering
\subfigure[mu detector 沉积能量]{\includegraphics[scale=0.2]{../result/detector_energy.png}}
\subfigure[mu detector 穿透距离]{\includegraphics[scale=0.2]{../result/detector_mm.png}}
\end{figure}

\end{frame}

\section{result}
\subsection{detector}
\begin{frame}
\begin{figure}[htbp]
\centering
\subfigure[mu detector 沉积能量除以穿透距离]{\includegraphics[scale=0.15]{../result/detector_en_mm.png}}
\subfigure[mu detector 衰变事例数]{\includegraphics[scale=0.16]{../result/detector_event.png}}
\end{figure}

\end{frame}

\section{result}
\subsection{PMT}
\begin{frame}
\begin{figure}[htbp]
\centering
\subfigure[PMT 读出光子信号 与他人对比]{\includegraphics[scale=0.3]{../result/pmt_con.png}}
\end{figure}

\end{frame}

\section{result}
\subsection{PMT}
\begin{frame}
\begin{figure}[htbp]
\centering
\subfigure[自己的物理过程的产生 PMT 信号的拟合]{\includegraphics[scale=0.4]{../result/pmt_my.png}}
\end{figure}

$$
    f(x) = a e^{-\frac{x}{b}}+c
$$
     Coefficients (with 95\% confidence bounds):
     \begin{align}
       a =&        1126  (1112, 1140)\nonumber\\
       b =&        1959  (1913, 2006)\nonumber\\
       c =&        1.84  (-3.07, 6.75)\nonumber\\
     \end{align}

\end{frame}

\section{result}
\subsection{PMT}
\begin{frame}
\begin{figure}[htbp]
\centering
\subfigure[他人的物理过程的产生 PMT 信号的拟合]{\includegraphics[scale=0.4]{../result/pmt_gear.png}}
\end{figure}

$$
    f(x) = a e^{-\frac{x}{b}}+c
$$
     Coefficients (with 95\% confidence bounds):
     \begin{align}
       a =&        1090  (1080, 1100)\nonumber\\
       b =&        2246  (2202, 2290)\nonumber\\
       c =&        -0.3383  (-4.785, 4.108)\nonumber\\
     \end{align}

\end{frame}

\section{展望}
\subsection{}
\begin{frame}
\begin{itemize}
  \item 增加光导结构
  \item 增加 PMT 复杂度,读取 PMT 产生的电子信号,模拟出实际信号
  \item 测量 $\mu$ 衰变出来的电子能谱
\end{itemize}
\end{frame}

\end{document}