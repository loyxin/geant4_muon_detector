本人于 2016 年参加唐健老师的本科生科研项目,期间得到唐老师的悉心指导,在此非常感谢唐老师,回首过往,感慨万千,亦师亦友,收获颇多,去年今时,盛情邀请,欣然许之,一年过去,也从小白变成了大白了。

本项目由国家基础科学人才培养基金资助 \subsection*{geant4}

Geant4是由欧洲核子中心(\+C\+E\+R\+N)和日本高能物理中心(\+K\+E\+K)主导开发的蒙特卡罗辐射输运计算通用程序包,主要应用在高能物理领域,可方便模拟强相互作用、弱相互作用等高能、超高能物理过程。

geant4安装与编写要点 \subsection*{缪子}

缪子是一种与电子相似的基本粒子,符号 \$$^\wedge$-\/\$, 它带有 1 单位负电荷,自旋为 1/2,质量为 105 \$\+MeV/c$^\wedge$2\$. 缪子的反粒子是 \$$^\wedge$+\$,拥有 1 单位正电荷.地球上绝大部分自然生成的缪子都由宇宙 线中的 \$\$ 介子产生(\$$^\wedge$-\/  $^\wedge$-\/ + \{\} \textbackslash{} $^\wedge$+ $^\wedge$+ + \$ ).大多数缪子在海平面以上 15km 处产生.因为不参与强相互作用,缪子的穿透能力很强.与绝大部分高能碰撞产生的粒子一样,缪子也是不稳定的.缪子的平均寿命约为\$2.\+2 s\$, 准确值为 2.\+1969811(22)×10−6 s,缪子的衰变方式\+: \$\$ $^\wedge$-\/  e$^\wedge$-\/ +  + \textbackslash{} $^\wedge$+  e$^\wedge$+ +  +  \$\$ \subsection*{实验装置}

探测器的核心是两个高 338.\+5 mm,下底面长143.5 mm , 上底面长 84 mm,厚 2 cm 的梯形形 塑 料 闪 烁 体,材 料 为 \$ C\+\_\+1\{0H\}\+\_\+\{14\} \$,其密 度为 1.\+25$\ast$g/cm3,它们间隔为 10 mm,当一个高能缪子或电子穿越探测器时,其轨迹上会产生微弱的荧光.这些荧光会被光电倍增管接收并放大为电信号.为了增加光的收集率,探测器被一层氧化铝反射层包裹

如图所示\+: 

{\bfseries 其中红色为探测器,白色为氧化铝反射层,绿色为光电倍增管的光窗}

$<$video id=\char`\"{}video\char`\"{} controls=\char`\"{}\char`\"{}$>$ $<$source id=\char`\"{}mpeg\char`\"{} src=\char`\"{}./pic/\+G\+Movie.\+mpeg\char`\"{} type=\char`\"{}video/mpeg\char`\"{}$>$ $<$/video$>$

\subsection*{运行结果}

在探测器板上方 75 mm 的高度发射\$$^\wedge$-\/\$,能量为 150 Me\+V, 方向为 z 轴负方向。 如图所示\+:   {\bfseries 其中红色径迹代表为带负电的缪子,绿色径迹代表是不带电的光子,缪子打在塑闪板上发出光子,光子在里面发射,直到打在光电倍增管上,被吸收}

\subsection*{文件介绍}

├── blank 存放一个空白的 geant4 代码,可以基于这个搭建自己的 geant4 程序

├── C\+Make\+Lists.\+txt cmake 文件

├── doc 注释生成文件 可以运行里面的 html 文件夹下的 index.\+html 文件查看文档,也可以打开 latex 文件夹下的 refman.\+pdf

├── pic 存放图片

├── macro 存放宏文件

│ ├── vis.\+mac 可视化图形界面的设定

│ ├── init\+\_\+vis.\+mac 初始化图形界面的宏文件,调用 vis.\+mac文件

│ ├── otherdrive.\+mac 调用不同图形界面引擎

│ ├── run.\+mac

├── shell\+\_\+py 存放shell 脚本和python数据处理文件

│ ├── detector

│ │ ├── cal\+\_\+aver\+\_\+var.\+py 读取 detect\+\_\+result .csv , 计算平均值和方差

│ │ ├── detect\+\_\+one\+\_\+csv.\+py 读取一个 \$\$ 的能量产生的数据,平均值和方差

│ │ ├── detect\+\_\+one\+\_\+csv.\+sh 通过运行 detect\+\_\+one\+\_\+csv.\+py,读取各个\$\$ 能量的数据合成一个 csv 文件

│ │ ├── detect.\+py 读取 muon\+\_\+nt\+\_\+detect.\+csv 文件,生成 detect\+\_\+result .csv 一个\$mu\$ 能量的1000事例的能量信息,位置信息

│ │ └── detect.\+sh 通过运行 detect.\+py,读取各个\$\$ 能量的数据,生成 detect\+\_\+result .csv

│ ├── pmt

│ │ ├── pmt\+\_\+cal.\+py 读取 pmt\+\_\+result.\+csv,统计衰变时间

│ │ ├── pmt\+\_\+cal.\+sh 通过运行 pmt\+\_\+cal.\+py, 统计各个\$\$ 能量衰变时间

│ │ ├── pmt.\+py 读取 muon\+\_\+nt\+\_\+pmt.\+csv 文件,生成 pmt\+\_\+result .csv 一个\$mu\$ 能量的1000事例的进入 pmt 的衰变光子能量平均值和方差以及光子个数,进入时间的平均值和方差,以及\$\$产生光子能量平均值和方差以及光子个数,进入时间的平均值和方差

│ │ └── pmt.\+sh 通过运行 pmt.\+py,读取各个\$\$ 能量的数据

│ └── run

│ ├── muon.\+sh 读取 run.\+mac 生成一个新的\$\$ 能量的宏文件 new.\+mac

│ └── test.\+sh 运行程序以及宏文件 new.\+mac, 并将生成的 muon\+\_\+nt\+\_\+detect.\+csv 和 muon\+\_\+nt\+\_\+pmt.\+csv 放在 data 文件夹下

├── result 存放运行结果

├── include 库文件的存放文件夹

├── src 源文件夹

├── \hyperlink{main_8cc}{main.\+cc} 程序主文件

├── Doxyfile doxygen 文件

Monday, 04. September 2017 09\+:53\+AM 